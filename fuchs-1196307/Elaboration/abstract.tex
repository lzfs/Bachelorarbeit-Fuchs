\vspace*{2cm}

\begin{center}
    \textbf{Kurzfassung}
\end{center}

\vspace*{1cm}

\noindent
Im Rahmen des \emph{HoPE-Projekts} wird das Verhalten von Menschen
bei der Nutzung von \emph{Ambient Displays} analysiert.
Tiefenkameras werden in der Forschung als eine mögliche Sensorik genutzt,
um Bewegungsabläufe aufzunehmen.
Konkret werden dazu Skelettdaten betrachtet.
Eine Analyse der Daten ist manuell schwer umsetzbar.
Daher muss ein methodisches Rahmenwerk entwickelt werden,
welches eine Evaluation von Sensordaten ermöglicht.
Diese Bachelorarbeit ist ein Beitrag zu diesem Rahmenwerk.
Ziel ist die Entwicklung eines Java-Tools,
welches derartige Daten ohne Vorwissen mithilfe deterministischer Algorithmen clustern kann.
Es kommt \emph{hierarchisches Clustering}
mithilfe des \emph{Dynamic Time Warping} Algorithmus zum Einsatz.
Es soll geklärt werden,
ob diese Algorithmen einen sinnvollen Beitrag zur Analyse von Bewegungsdaten leisten können.
Nach der Implementierung erfolgt ein Clustering von \emph{Kinect-Bewegungsdaten}.
Mithilfe einer Ground-Truth-Analyse und deskriptiven Statistiken
wird die Qualität der gefundenen Cluster überprüft.
Insbesondere hat hat sich dabei herausgestellt, dass ein geeigneter Threshold von großer Bedeutung ist.
Die Evaluation bestätigt, dass die verwendeten Algorithmen zum Clustering
von Time-Series Daten genutzt werden können.