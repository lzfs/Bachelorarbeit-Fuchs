\vspace*{2cm}

\begin{center}
    \textbf{Abstract}
\end{center}

\vspace*{1cm}

\noindent
Im Rahmen des \emph{HoPE-Projekts} wird das Verhalten von Menschen
bei der Nutzung von \emph{Ambient Displays} analysiert.
Dazu soll auch ein methodisches Rahmenwerk entwickelt werden,
welches eine Evaluation von Sensordaten ermöglicht.
Tiefenkameras werden seit vielen Jahren in der Forschung genutzt,
um Bewegungsabläufe aufzunehmen.
Solche Bewegungsdaten sind sogenannte Time-Series Data.
Ziel dieser Arbeit ist es ein Software-Tool zu entwickeln,
welches derartige Datensätze clustern kann.
Dazu soll \emph{hierarchisches Clustering}
mithilfe des \emph{Dynamic Time Warping Algorthmus} zum Einsatz kommen.
Die Wahl der Algorithmen wird begründet.
Zudem erfolgt eine ausführliche Beschreibung dieser.
Anschließend wird die Konzeption und Implementierung des Tools dargestellt.
Die Software wurde dabei so allgemein gestaltet,
dass sie für verschiedene Datensätze verwendet werden kann.
Auch mögliche Limitierungen werden beschrieben.
Abschließend erfolgt ein Clustering von \emph{Kinect-Bewegungsdaten}.
Mithilfe einer Ground-Truth-Analyse und deskriptiven Statistiken
wird die Qualität der gefundenen Cluster überprüft.
Diese Analyse zeigt, dass das entwickelte Tool zum Clustering
von Time-Series Data genutzt werden kann.