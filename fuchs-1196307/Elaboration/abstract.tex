\vspace*{2cm}

\begin{center}
    \textbf{Abstract}
\end{center}

\vspace*{1cm}

\noindent
Im Rahmen des \emph{HoPE-Projekts} wird das Verhalten von Menschen
bei der Nutzung von \emph{Ambient Displays} analysiert.
Dazu soll ein methodisches Rahmenwerk entwickelt werden,
welches eine Evaluation von Sensordaten ermöglicht.
Tiefenkameras sind eine mögliche Sensorik.
Sie werden seit vielen Jahren in der Forschung genutzt,
um Bewegungsabläufe aufzunehmen.
Solche Bewegungsdaten sind sogenannte Time-Series Data.
Ziel dieser Arbeit ist die Entwicklung eines Software-Tools,
welches derartige Daten clustern kann.
Dazu soll \emph{hierarchisches Clustering}
mithilfe des \emph{Dynamic Time Warping}-Algorthmus zum Einsatz kommen.
Neben der Begründung für die Wahl der Algorithmen,
erfolgt zudem eine Beschreibung dieser.
Des Weiteren wird die Konzeption und Implementierung des Tools dargestellt.
Die Software wird dabei so generisch gestaltet,
dass sie auf das Format verschiedener Datensätze angepasst werden kann.
Auch mögliche Limitierungen werden beschrieben.
Abschließend erfolgt ein Clustering von \emph{Kinect-Bewegungsdaten}.
Mithilfe einer Ground-Truth-Analyse und deskriptiven Statistiken
wird die Qualität der gefundenen Cluster überprüft.
Diese Evaluation bestätigt, dass das entwickelte Tool zum Clustering
von Time-Series Data genutzt werden kann.