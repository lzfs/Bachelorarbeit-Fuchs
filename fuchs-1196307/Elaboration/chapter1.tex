\chapter{Einleitung und Motivation}
\label{chapter1}
Von September 2021 bis August 2024 forschen die Hochschule für angewandte Wissenschaften Hamburg
und die Universität der Bundeswehr München im Rahmen des \emph{HoPE-Projekts} zusammen an Effekten
rund um die Steigerung der Aufmerksamkeit bei der Nutzung von großen, interaktiven Wandbildschirmen,
sogenannten \emph{Ambient Displays}.
Dieser Bereich weist noch einen grundsätzlichen Forschungsbedarf auf.
Im Projekt wird unter anderem der \emph{Honeypot-Effekt} erforscht \citep{unibw_honeypot-effekt_2021}.
Dieser beschreibt in der \ac{HCI} wie Menschen, die mit einem System interagieren,
weitere Passanten anregen die Interaktion zu beobachten oder sogar an ihr teilzuhaben \citep{wouters_uncovering_2016}.
Der \emph{Honeypot-Effekt} soll bei der Nutzung von \emph{Ambient Displays} im öffentlichen
und halb-öffentlichen Raum in Langzeit-Feldstudien analysiert werden.
Dabei soll auch der Aspekt der Datenerhebung und -analyse weiter ausgebaut werden.
Hierzu muss ein methodisches Rahmenwerk entwickelt werden, welches eine auf Sensordaten-basierende,
automatische und zeitlich uneingeschränkte Evaluation von \emph{Ambient Displays} ermöglicht \citep{unibw_honeypot-effekt_2021}.
Die Sensordaten werden von Body-Tracking-Kameras bereitgestellt.
Es liegt ein Datensatz vor bei dem Wandbildschirme mit Microsoft Kinect v2 Kameras ausgestattet wurden.
Diese zeichnen die Interaktion von Nutzern mit den Displays auf,
wodurch das Nutzerverhalten zu einem späteren Zeitpunkt ausgewertet werden kann.
Die Daten liegen als \ac{TSD} vor.
Bei diesem Datentyp handelt es sich um geordnete Sequenzen von Datenpunkten,
die über eine gewisse Zeit hinweg aufgenommen werden \citep{ali_clustering_2019}.
\ac{TSD} enthalten oft wichtige Informationen, die durch eine Sortierung entdeckt werden können.
Dabei stellt sich die Frage, ob sich eine Menge von Kategorien identifizieren lässt,
die etwas über das Verhalten von Menschen vor Wandbildschirmen aussagt.
Bei großen Datensätzen ist eine manuelle Evaluation zur Beantwortung der Frage nicht möglich.
Daher muss die Sortierung automatisiert werden.
In dieser Bachelorarbeit wird versucht, dieses Problem mithilfe von deterministischen Algorithmen zu lösen.
Wesentliches Ziel ist die Implementierung eines Systems zur Kategorisierung der vorliegenden Kinect-Bewegungsdaten.
Eine gängige Methode zur Analyse sind dabei \emph{Clustering}-Verfahren \citep{aghabozorgi_time-series_2015}.
Konkret soll \emph{hierarchisches Clustering} mithilfe des \emph{\ac{DTW}}-Algorithmus eingesetzt werden.

Dabei ist die Arbeit wie folgt aufgebaut:
Zunächst wird in \autoref{chapter2} auf wichtige Grundlagen für diese Arbeit eingegangen.
Mögliche Modelle der Interaktion mit Wandbildschirmen
werden in \autoref{2-ModelleInteraktion-Wandbildschirme} vorgestellt.
Das entwickelte Tool soll später dazu beitragen, konkrete Bewegungsdaten in solche Modelle einzuteilen.
Außerdem werden die Grundzüge der Kinect Kamera und des vorliegenden Datensatzes dargestellt
(\autoref{2-SpezifikationKinect}, \autoref{2-StrukturDatensatz}).
In \autoref{2-Problembeschreibung} erfolgt eine tiefergehende Erläuterung der Problemstellung.
\autoref{chapter3} widmet sich dem Hierarchischen Clustering und dem \ac*{DTW}-Algorithmus.
Die Wahl dieser Algorithmen wird hier zudem begründet.
Nach einer ausführlichen Beschreibung der Funktionsweise (\autoref{3-Clustering}, \autoref{3-DTW})
wird das Vorgegen in \autoref{3-Example} an einem Beispiel veranschaulicht.
Abschließend wird auf verwandte Literatur (\autoref{3-RelatedWork})
und den Einsatz im Kontext von Kinect-Bewegungsdaten (\autoref{3-Einsatz}) eingegangen.
\autoref{chapter4} behandelt die Konzeption des Software-Tools.
Dafür werden in \autoref{4-Anforderungsanalyse} die Anforderungen betrachtet,
bevor im Rest des Kapitels auf den geplanten Programmablauf,
sowie nötige Teilsysteme eingegangen wird.
Anschließend erfolgt die eigentliche Implementierung der Software.
In \autoref{chapter5} wird der Aufbau beleuchtet.
Außerdem werden zentrale Ausschnitte des Codes
und Abweichungen zur Konzeption beschrieben.
Abschließend erfolgt eine Evaluation (\autoref{chapter6}).
Zunächst wird hier auf Limitierungen der Anwendung
und die zur Evaluation verwendeten Daten eingegangen.
Diese werden mithilfe des Tools geclustert.
Die Qualität der Cluster wird in \autoref{6-GroundTruth} mithilfe von Ground-Truth Daten
und in \autoref{6-Statistical} durch deskriptive Statistiken überprüft.
Ein abschließendes Fazit erfolgt in \autoref{chapter7}.