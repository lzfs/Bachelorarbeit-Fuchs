\chapter{Einleitung und Motivation}
\label{chapter1}

\section{Einleitung}
\label{1-Einleitung}


\section{Motivation}
\label{1-Motivation}
Von September 2021 bis August 2024 forschen die Hochschule für angewandte Wissenschaften in Hamburg
und die Universität der Bundeswehr München im Rahmen des \emph{HoPE-Projekts} zusammen an Effekten
rund um die Steigerung der Aufmerksamkeit bei der Nutzung von großen, interaktiven Wandbildschirmen,
sogenannten \emph{Ambient Displays}.
Dieser Bereich weist noch einen grundsätzlichen Forschungsbedarf auf.
Im Projekt wird unter anderem der \emph{Honeypot-Effekt} erforscht \citep{unibw_honeypot-effekt_2021}.
Dieser beschreibt in der \ac{HCI} wie Menschen, die mit einem System interagieren,
weitere Passanten anregen die Interaktion zu beobachten oder sogar an ihr teilzuhaben \citep{wouters_uncovering_2016}.
Der \emph{Honeypot-Effekt} soll bei der Nutzung von \emph{Ambient Displays} im öffentlichen
und halb-öffentlichen Raum in Langzeit-Feldstudien analysiert werden.
Dabei soll auch der Aspekt der Datenerhebung und -analyse weiter ausgebaut werden.
Hierzu muss ein methodisches Rahmenwerk entwickelt werden, welches eine auf Sensordaten-basierende,
automatische und zeitlich uneingeschränkte Evaluation von \emph{Ambient Displays} ermöglicht \citep{unibw_honeypot-effekt_2021}.
Die Sensordaten werden von Body-Tracking-Kameras bereitgestellt.
Konkret wurden die Wandbildschirme im vorliegenden Datensatz mit Microsoft Kinect v2 Kameras ausgestattet.
Diese zeichnen die Interaktion von Nutzern mit den Displays auf,
wodurch das Nutzerverhalten zu einem späteren Zeitpunkt ausgewertet werden kann.
Die Daten liegen als \ac{TSD} vor.
Bei diesem Datentyp handelt es sich um geordnete Sequenzen von Datenpunkten,
die über eine gewisse Zeit hinweg aufgenommen werden \citep{ali_clustering_2019}.
\ac{TSD} enthalten oft wichtige Informationen die durch Mustererkennung entdeckt werden können.
Eine gängige Methode zur Analyse sind hierbei \emph{Clustering}-Verfahren \citep{aghabozorgi_time-series_2015}.
Lässt sich eine Menge von Kategorien identifizieren, die etwas über das Verhalten von Menschen vor Wandbildschirmen aussagt?
Ist die Einteilung in diese Kategorien automatisierbar?
Aufgrund der Größe des vorliegenden Datensatzes ist eine manuelle Evaluation zur Beantwortung der Fragen nicht möglich.
In dieser Bachelorarbeit wird daher versucht, dieses Problem mithilfe von deterministischen Algorithmen zu lösen.
Wesentliches Ziel ist die Implementierung eines Systems zur Kategorisierung der vorliegenden Kinect-Bewegungsdaten.
Dieses System wird detailiert beschrieben und anschließend evaluiert.
Dabei soll \emph{hierarchisches Clustering} mithilfe des \emph{\ac{DTW}}-Algorithmus eingesetzt werden.
Außerdem wird die Frage beantwortet, welches Vorwissen benötigt wird,
um das Verfahren einsetzen zu können und welche Attribute zur Analyse interessant sind (z.B. Laufpfade oder Engaged-Werte).